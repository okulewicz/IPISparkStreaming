\documentclass{beamer}
\usepackage[utf8]{inputenc}
\usepackage{graphicx}
\usepackage{amsmath}
\usepackage{amsfonts}
\usepackage{amssymb}
\usepackage{float}
\usepackage[verbose]{wrapfig}
\usepackage{textcomp}

\newcommand{\argmax}{\operatornamewithlimits{argmax}}

\mode<presentation>
{
  \usetheme{Warsaw}
  \useinnertheme{circles}
  \usecolortheme{crane}

  \setbeamercovered{transparent}
  \beamertemplatenavigationsymbolsempty
  \useoutertheme{infolines}
}

\title[Spark Streaming] % (optional, use only with long paper titles)
{Spark Streaming}

\author %[Author, Another] % (optional, use only with lots of authors)
{Mateusz, Kope{\'c}, Micha{\l} Okulewicz}

\institute %[Universities of Somewhere and Elsewhere] % (optional, but mostly needed)
{
Institute of Computer Science\\
Polish Academy of Sciences}

\date %[ ] % (optional, should be abbreviation of conference name)
{Big Data \\27 November 2014}

\subject{Real-time distributed computing}

\begin{document}

\begin{frame}
  \titlepage
\end{frame}

\begin{frame}
  \frametitle{Presentation Plan}
  \tableofcontents
  % You might wish to add the option [pausesections]
\end{frame}

\section{Introduction}

\subsection*{Apache Spark\texttrademark}
\begin{frame}
\frametitle{What is Apache Spark\texttrademark?}

\begin{block}{Apache Spark\texttrademark}
\begin{itemize}
	\item distributed computations system
	\item not only MapReduce applications
	\item supports in-memory operations
	\item may use HDFS
\end{itemize}

\end{block}

\end{frame}

\begin{frame}
\frametitle{What is Spark Streaming?}

\begin{block}{Spark Streaming}
\begin{itemize}
	\item subproject of Apache Spark\texttrademark
	\item allows for real-time distributed stream processing
	\item utilizes an idea called Discretized Stream (DStream)
\end{itemize}

\end{block}

\end{frame}


\subsection*{Real-time computing}
\begin{frame}
\frametitle{Real-time computing}
	\begin{itemize}
		\item Fraud detection
		\item Volcano eruptions / earthquakes detection
	\end{itemize}
\end{frame}

\section{Spark architecture}
\begin{frame}
\frametitle{Spark architecture}
\end{frame}

\section{Example stream processing task}
\begin{frame}
\frametitle{Example stream processing task}
\end{frame}

\section{How to run Spark Streaming?}
\begin{frame}
\frametitle{How to run Spark?}
\begin{block}{Spark Streaming}
	\begin{itemize}
		\item Get precompiled Spark 1.1.0 for Hadoop 1.x from \texttt{/home/2012/m.okulewicz/spark} and unpack it
		\item If necessary edit: \texttt{conf/spark-env.sh} and add location of~\texttt{JAVA\_HOME}
		\item Run \texttt{sbin/start-slave.sh 1 spark://phd03.phd.ipipan.waw.pl}
		\item Check in browser if \texttt{http://localhost:8081} is available and master points to \texttt{phd03.phd.ipipan.waw.pl}
	\end{itemize}
	\end{block}
\end{frame}

\begin{frame}[allowframebreaks]
	\frametitle<presentation>{Bibliography}
	\nocite{*}
	\bibliographystyle{plain}
	\tiny
	\bibliography{SparkStream}
\end{frame}

\end{document}


